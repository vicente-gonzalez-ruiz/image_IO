% Emacs, this is -*-latex-*-

\title{Image IO}

\maketitle
\tableofcontents

\section{Gray-scale images}
%{{{

Gray-Scale (GS) images have only one (usually
\href{https://en.wikipedia.org/wiki/Luminance}{luminance}, expressed
proportionally by the gray colors) channel\footnote{They are usually
represented in color displays using gray tones (colors where the same
amount of red, green and blue components have been mixed). However, it
is quite difficult to find a color display with more than 256 gray
colors.}. In most of the cases, the number of bits per pixel (the so
called ``depth'' of the image) is $8$, but it is not rare to use up to
$16$ bits/pixel in specific contexts such as in
\href{https://en.wikipedia.org/wiki/Medical_imaging}{Medical Imaging}
and \href{https://en.wikipedia.org/wiki/Remote_sensing}{Remote
  Sensing}~\cite{burger2016digital}.
%}}}

\section{$\text{RGB}$ images}

$\text{RGB}$ (Red, Green, and Blue) is an additive color system, which
means that all colors ``start'' with black and are created by adding
some intensity of the primary colors red, green and
blue~\cite{burger2016digital}. In $\text{RGB}$ images, the color of a
pixel depends on the
\href{https://en.wikipedia.org/wiki/Visible_spectrum}{frequency of the
  light that the pixel represents}. Such information can be
represented in a number of different encoding systems known as
\href{https://en.wikipedia.org/wiki/Color_space}{color spaces}. Among
all those systems, the $\text{RGB}$ color space is the most used
because $\text{RGB}$ images can be obtained directly from the light
signal using color filters.\footnote{Specifically, a red (R) filter, a
green (G) filter and a blue (B) filter.}

\section{Resources}

\renewcommand{\addcontentsline}[3]{} % Remove functionality of \addcontentsline
\bibliography{image-processing}
